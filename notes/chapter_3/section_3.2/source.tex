\section{Solutions of Linear Homogeneous Equations; the Wronskian}
    Let $p$ and $q$ be continuous functions on an open interval $I = (\alpha, \beta)$ where $\alpha$ and $\beta$ can be anything including $\infty$. Then, for any function $\phi$ that is twice differentiable on $I$,
    \begin{equation*}
        L[\phi] = \phi'' + p\phi' + q\phi
    \end{equation*}
    So,
    \begin{equation*}
        L[\phi](t) = \phi''(t) + p(t)\phi'(t) + q(t)\phi(t)
    \end{equation*}
    The operator $L$ is often written as $L = D^2 + pD + q$, where $D$ is the derivative operator.
    \newline \indent
    So the initial value problem is
    \begin{equation*}
        L[y] = y'' + p(t)y' + q(t)y
    \end{equation*}
    \begin{equation*}
        y(t_0) = y_0 \qquad y'(t_0) = y'_0
    \end{equation*}
    \begin{theorem}
        Consider the initial value problem
        \begin{equation*}
            y'' + p(t)y' + q(t)y = g(t), \qquad y(t_0) = y_0, \qquad y'(t_0) = y'_0
        \end{equation*}
        where $p$, $q$, and $g$ are continuous on an open interval $I$ that contains the point $t_0$. Then there is exactly one solution $y = \phi(t)$ of this problem, and the solution exists throughout the interval $I$.
    \end{theorem}
    This theorem says three things:
    \begin{enumerate}
        \item A solution \textit{exists}
        \item The solution is \textit{unique}
        \item The solution $\phi$ is defined \textit{throughout the interval} $I$ where the coefficients are continuous and is at least twice differentiable there.
    \end{enumerate}
    \indent For most second order problems, we cannot write a useful expression for the solution. This is a major difference between first order and second order linear equations.
    \begin{theorem}[Principle of Superposition]
        If $y_1$ and $y_2$ are two solutions fo the differential equation,
        \begin{equation*}
            L[y] = y'' + p(t)y' + q(t)y = 0
        \end{equation*}
        then the linear combination $c_1y_1 + c_2y_2$ is also a solution for any values of the constants $c_1$ and $c_2$
    \end{theorem}
    \begin{proof}
        To prove this, we substitute $$y = c_1y_1(t) + c_2y_2(t)$$ for $y$.
        \begin{align*}
            L[c_1y_1 + c_2y_2] = [c_1y_1 + c_2y_2]'' + p[c_1y_1 + c_2y_2]' + q[c_1y_1 + c_2y_2] \\
            = c_1y_1'' + c_2y_2'' + c_1py_1' + c_2py_2' + c_1qy_1 + c_2qy_2 \\
            = c_1[y_1'' + py_1' + qy_1] + c_2[y_2'' + py_2' + qy_2] \\
            = c_1L[y_1] + c_2L[y_2] = c_1(0) + c_2(0) = 0
        \end{align*}
        since $L[y_1] = L[y_2] = 0$ because they are both solutions.
    \end{proof}
    This theorem essentially states that beginning with two solutions, we can construct an infinite family of solutions. Now, to address if all solutions of the equation are included in this family. First, we find constants to match our intial values.
    $$c_1y_1(t_0) + c_2y_2(t_0) = y_0$$
    $$c_1y_1'(t_0) + c_2y_2'(t_0) = y_0'$$
    which can be written as
    \begin{equation*}
        \begin{bmatrix}
            y_1(t_0) & y_2(t_0) \\
            y_1'(t_0) & y_2'(t_0)
        \end{bmatrix}
        \begin{bmatrix}
            c_1 \\
            c_2
        \end{bmatrix}
        =
        \begin{bmatrix}
            y_0 \\
            y_0'
        \end{bmatrix}
    \end{equation*}
    The determinant of the coefficients of the system
    \begin{equation*}
        W = \begin{vmatrix}
            y_1(t_0) & y_2(t_0) \\
            y_1'(t_0) & y_2'(t_0)
        \end{vmatrix} = y_1(t_0)y_2'(t_0) - y_1'(t_0)y_2(t_0)
    \end{equation*}
    If $W \neq 0$,
    \begin{equation*}
        c_1 = \frac{y_0y_2'(t_0) - y_0'y_2(t_0)}{y_1(t_0)y_2'(t_0) - y_1'(t_0)y_2(t_0)}, \qquad c_2 = \frac{-y_0y_1'(t_0) + y_0'y_1(t_0)}{y_1(t_0)y_2'(t_0) - y_1'(t_0)y_2(t_0)}
    \end{equation*}
    or 
    \begin{equation*}
        c_1 = \frac{
            \begin{vmatrix}
                y_0 & y_2(t_0) \\
                y_0' & y_2'(t_0)
            \end{vmatrix}
        }{
            \begin{vmatrix}
                y_1(t_0) & y_2(t_0) \\
                y_1'(t_0) & y_2'(t_0)
            \end{vmatrix}
        }, \qquad 
        c_2 = \frac{
            \begin{vmatrix}
                y_1(t_0) & y_0 \\
                y_1'(t_0) & y_0'
            \end{vmatrix}
        }{
            \begin{vmatrix}
                y_1(t_0) & y_2(t_0) \\
                y_1'(t_0) & y_2'(t_0)
            \end{vmatrix}
        }
    \end{equation*}
    These values of $c_1$ and $c_2$ satisfy the initial conditions and differential equation. But this is only if $W \neq 0$. If $W = 0$ then there are no solutions unless the numerators are equal to 0. Since otherwise the initial cannot be satisfied no matter what constants are chosen.
    \newline \indent
    The determinant $W$ is called the \textbf{Wronskian determinant}, or simply the \textbf{Wronskian}, of the solutions $y_1$ and $y_2$. Sometimes we use $W(y_1, y_2)(t_0)$.
    \begin{theorem}
        Suppose that $y_1$ and $y_2$ are two solutions of
        \begin{equation*}
            L[y] = y'' + p(t)y' + q(t)y = 0
        \end{equation*}
        and that the initial conditions
        \begin{equation*}
            y(t_0) = y_0 \qquad y'(t_0) = y'_0
        \end{equation*}
        are assigned. Then it is always possible to choose the constants $c_1, c_2$ so that 
        \begin{equation*}
            y = c_1y_1(t) + c_2y_2(t)
        \end{equation*}
        satisfies the differential equation and the initial conditions if and only if the Wronskian
        \begin{equation*}
            W = y_1y_2' - y_1'y_2
        \end{equation*}
        is not zero at $t_0$.
    \end{theorem}
    \begin{theorem}
        Suppose that $y_1$ and $y_2$ are two solutions of the differential equation,
        \begin{equation*}
            L[y] = y'' + p(t)y' + q(t)y = 0
        \end{equation*}
        Then the family of solutions
        \begin{equation*}
            y = c_1y_1(t) + c_2y_2(t)
        \end{equation*}
        with arbitrary coefficients $c_1$ and $c_2$ includes every solution of the equation if and only if there is a point $t_0$ where the Wronskian of $y_1$ and $y_2$ is not zero.
    \end{theorem}
    \begin{proof}
        Let $\phi$ be any arbitrary solution of the equation. To prove this, we show that $\phi$ as a part of the family $c_1y_1(t) + c_2y_2(t)$. Let $t_0$ be a point be point where the Wronskian is nonzero. So,
        \begin{equation*}
            y_0 = \phi(t_0), \qquad y_0' = \phi'(t_0)
        \end{equation*}
        Now, we consider the initial value problem
        \begin{equation*}
            y'' + p(t)y' + q(t)y = 0, \quad y(t_0) = y_0, \quad y'(t_0) = y'_0
        \end{equation*}
        Since the Wronskian is nonzero, it is possible to choose a $c_1$ and $c_2$, so that 
        \begin{equation*}
            \phi(t) = c_1y_1(t) + c_2y_2(t) 
        \end{equation*}
        so $\phi$ is included in the family. If there is no point $t_0$ where the Wronskian is nonzero, there are values $y_0$ and $y_0'$ that the system has no solutions for $c_1$ and $c_2$. Select a pair of such values and choose the solution $\phi(t)$ that satisfies the initial condition. So, if the $W = 0$, then there exists a solution that does not fit in the family.
    \end{proof}
    \begin{equation*}
        y = c_1y_1(t) + c_2y_2(t)
    \end{equation*}
    with arbitrary constant coefficients is the \textbf{general solution}. And the solutions $y_1$ and $y_2$ are said to form a \textbf{fundamental set of solutions} if and only if their Wronskian is nonzero.
    \begin{theorem}
        Consider the differential equation
        \begin{equation*}
            L[y] = y'' + p(t)y' + q(t)y = 0
        \end{equation*}
        whose coefficients $p$ and $q$ are continuous on some open interval $I$. Choose some point $t_0$ in $I$. Let $y_1$ be the solution of the equation that also satisfies the initial conditions
        $$y(t_0) = 1, \qquad y'(t_0) = 0$$
        and let $y_2$ be the solutions of the equation that satisfies the initial conditions
        $$y(t_0) = 0, \qquad y'(t_0) = 1$$
        Then $y_1$ and $y_2$ form a fundamental set of solutions.
    \end{theorem}
    \begin{proof}
        The \textit{existence} of $y_1$ and $y_2$ is ensured by the existence part of theorem 3.2.1, and they form a fundamental set of solutions since
        \begin{equation*}
            W(y_1, y_2)(t_0) = \begin{vmatrix}
                y_1(t_0) & y_2(t_0) \\
                y_1'(t_0) & y_2'(t_0)
            \end{vmatrix} = \begin{vmatrix}
                1 & 0 \\
                0 & 1
            \end{vmatrix} = 1 \neq 0
        \end{equation*}
    \end{proof}
    \begin{theorem}
        Consider the differential equation
        \begin{equation*}
            L[y] = y'' + p(t)y' + q(t)y = 0
        \end{equation*}
        where $p$ and $q$ are continuous real-valued functions. If $y = u(t) + iv(t)$ is a complex-valued solution, then its real part $u$ and imaginary part $v$ are both also solutions.
    \end{theorem}
    \begin{proof}
        To prove this we substitute $y = u(t) + iv(t)$ in $L[y]$,
        \begin{align*}
            L[y] = u''(t) + iv''(t) + p(t)[u'(t) + iv'(t)] + q(t)[u(t) + iv(t)] \\
            = u''(t) + p(t)u'(t) + q(t)u(t) + i[v''(t) + p(t)v'(t) + q(t)v(t)] \\
            = L[u(t)] + iL[v(t)] = 0
        \end{align*}
        For a complex-number to be zero both the real and imaginary parts have to be zero, so both $L[u(t)] = 0$ and $L[v(t)] = 0$, and are both solutions.
    \end{proof}
    Also, the complex conjugate $\bar{y}$ of a solution $y$ is also a solution since it is a linear combination of $u(t)$ and $v(t)$.
    \begin{theorem}[Abel's Theorem]
        If $y_1$ and $y_2$ are solutions to the differential equation
        \begin{equation*}
            L[y] = y'' + p(t)y' + q(t)y = 0
        \end{equation*}
        where $p$ and $q$ are continuous on an open interval $I$, then the Wronskian $W$ is given by
        \begin{equation*}
            W(y_1, y_2)(t) = c\exp[-\int p(t)dt]
        \end{equation*}
        where $c$ is a constant dependent on $y_1$ and $y_2$ but not $t$. So, $W$ is zero for all $t \in I$ if $c = 0$, else it is nonzero for all $t \in I$.         
    \end{theorem}
    \begin{proof}
        We know
        $$y_1'' + p(t)y_1' + q(t)y_1 = 0$$
        $$y_2'' + p(t)y_2' + q(t)y_2 = 0$$
        if we multiply the first equation by $-y_2$ and the second with $y_1$,a dn add the resulting equations, we get
        $$(y_1y_2'' - y_1''y_2) + p(t)(y_1y_2' - y_1'y_2) = 0$$
        We see that 
        $$W' = y_1y_2'' - y_1''y_2$$
        so
        $$W' + p(t)W = 0$$
        which we can solve to get
        \begin{equation*}
            W(y_1, y_2)(t) = c\exp[-\int p(t)dt]
        \end{equation*}
        and since the exponential function cannot be 0, $W = 0$ if and only if $c = 0$.
    \end{proof}
