\section{Solutions of Some Differential Equations}
    If a differential equation $dy/dx = f(x, y)$ can be written as 
    \begin{equation*}
        M(x) + N(y)\frac{dy}{dx} = 0
    \end{equation*}
    then it is said to be \textbf{separable} since it can be written as $$M(x)dx + N(y)dy = 0$$ and integrated.
    \newline \indent
    Let
    \begin{equation*}
        H'_1(x) = M(x) \qquad H'_2(y) = M(y)
    \end{equation*}
    so then the previous equation becomes
    \begin{equation*}
        H'_1(x) + H'_2(y)\frac{dy}{dx} = 0
    \end{equation*}
    By the chain rule:
    \begin{equation*}
        H'_2(y)\frac{dy}{dx} = \frac{d}{dy}H_2(y)\frac{dy}{dx} = \frac{d}{dx}H_2(y)
    \end{equation*}
    which gives us
    \begin{equation*}
        \frac{d}{dx}[H_1(x) + H_2(y)] = 0
    \end{equation*}
    By integrating, we get
    \begin{equation*}
        H_1(x) + H_2(y) = c
    \end{equation*}
    Any differentiable function $y = \phi(x)$ that satisfies $H_1(x) + H_2(y) = c$ is a solution of the original differentiable equation. The differential equation and the initial condition $y(x_0) = y_0$ forms an initial value problem. We can use the initial value to find the correct c:
    \begin{equation*}
        c = H_1(x_0) + H_2(y_0)
    \end{equation*}
    so
    \begin{equation*}
        c = H_1(x_0) + H_2(y_0) = H_1(x) + H_2(y)
    \end{equation*}
    \begin{equation*}
        (H_1(x) - H_1(x_0)) + (H_2(y) - H_2(y_0)) = 0
    \end{equation*}
    \begin{equation*}
        \int_{x_0}^x M(s) ds + \int_{y_0}^y N(s) ds = 0
    \end{equation*}
    since 
    \begin{equation*}
        H_1(x) - H_1(x_0) = \int_{x_0}^x M(s) ds \qquad H_2(y) - H_2(y_0) = \int_{y_0}^y N(s) ds
    \end{equation*}
    \indent
    \textit{Note 1:} Sometimes the solution to $$\frac{dy}{dx} = f(x, y)$$ has a constant solution $y = y_0$, which occurs when $f(x, y_0) = 0$ for all $x$ and for $y_0$. For example,
    \begin{equation*}
        \frac{dy}{dx} = \frac{(y - 3)\cos x}{1 + 2y^2}
    \end{equation*}
    has a solution $y = 3$.
    \newline \indent
    \textit{Note 2:} Sometimes if a function is non-linear it helps to regard both $x$ and $y$ as functions of a third variable $t$.
    \begin{equation*}
        \frac{dy}{dt} = \frac{dy/dt}{dx/dt}
    \end{equation*}
    \indent
    \textit{Note 3:} Sometimes it is not easy to solve explicitly for $y$ as a function of $x$. In these cases, it is better to leave the solution in implicit form.
    