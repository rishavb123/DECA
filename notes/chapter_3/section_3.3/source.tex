\section{Complex Roots of the Characteristic Equation}
    We continue our discussion of the equation
    \begin{equation*}
        ay'' + by' + cy = 0
    \end{equation*}
    where $a$, $b$, and $c$ are given real numbers. If we seek solutions of the form $y = e^{rt}$, the $r$ must be a root of the characteristic equation
    \begin{equation*}
        ar^2 + br + c = 0
    \end{equation*}
    We previously showed that if $r_1$ and $r_2$ are real and different, which occurs if $b^2 - 4ac > 0$, then the general solution of our differential equation is
    \begin{equation*}
        y = c_1e^{r_1t} + c_2e^{r_2t}
    \end{equation*}
    Now if $b^2 - 4ac < 0$, then the roots of the characteristic equation are conjugate complex numbers,
    $$r_1 = \lambda + i\mu, \qquad r_2 = \lambda - i\mu$$
    where $\lambda$ and $\mu$ are real. The corresponding expressions for $y$ are
    $$y_1(t) = \exp[(\lambda + i\mu)t], \qquad y_2(t) = \exp[(\lambda - i\mu)t]$$
    What does it mean to raise the number $e$ to a complex power.
    \textbf{Euler's Formula.} A Taylor series for $e^t$ about $t=0$ is
    $$e^t = \sum_{n=0}^\infty \frac{t^n}{n!}$$
    If we replace $t$ with $it$,
    \begin{align*}
        e^{it} = \sum_{n=0}^\infty \frac{i^nt^n}{n!} \\
        = \sum_{n=0}^\infty \frac{(-1)^nt^{2n}}{(2n)!} + i\sum_{n=0}^\infty \frac{(-1)^{n-1}t^{2n-1}}{(2n-1)!}
    \end{align*}
    The real portion of the expression is the Taylor expansion for $\cos t$, and the imaginary part is the Taylor expansion for $\sin t$, so
    $$e^{it} = \cos t + i\sin t$$
    which is known as Euler's formula.
    $$e^{-it} = \cos t - i\sin t$$
    since $\cos(-t) = \cos t$ and $\sin(-t) = -\sin t$. In general if $t$ is replaced with $\mu t$
    $$e^{i\mu t} = \cos \mu t + i\sin \mu t$$
    Now if we have a complex number $\lambda + i\mu$
    \begin{align*}
        e^{(\lambda + i\mu)t} = e^{\lambda t}e^{i\mu t}
        = e^{\lambda t}(\cos \mu t + i\sin \mu)
        = e^{\lambda t}\cos \mu t + ie^{\lambda t}\sin \mu t
    \end{align*}
    \begin{equation*}
        \frac{d}{dt}(e^{rt}) = re^{rt}
    \end{equation*}
    still holds true with complex $r$s and we can verify it using the definition above.
    \newline
    \textbf{Complex Roots; The General Case.} If we have two solutions for $y$, $e^{\lambda t}\cos \mu t \pm ie^{\lambda t}\sin \mu t$, we can use the real and imaginary parts as a fundamental set of solutions since
    \begin{equation*}
        W(u, v)(t) = \mu e^{2\lambda t}
    \end{equation*}
    by direct computation. $W \neq 0$ as long as $\mu \neq 0$, so $u$ and $v$ form a fundamental set. If $\mu$ is zero then the roots are real and we already talked about that. So the general solution of the original equation is
    \begin{equation*}
        y = c_1e^{\lambda t}\cos \mu t + c_2 e^{\lambda t} \sin \mu t
    \end{equation*}
    
    