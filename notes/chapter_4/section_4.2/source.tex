\section{Homogeneous Equations with Constant Coefficients}
    Consider the $n$th order linear homogeneous differential equation
    $$L[y] = a_0y^{(n)} + a_1y^{(n-1)} + \dots + a_{n-1}y' + a_ny = 0$$
    $$L[e^{rt}] = e^{rt}(a_0r^n + a_1r^{n-1} + \dots + a_{n-1}r + a_n) = e^{rt}Z(r)$$
    for all $r$, where
    $$Z(r) = a_0r^n + a_1r^{n-1} + \dots + a_{n-1}r + a_n$$
    For those values of $r$ for which $Z(r) = 0$, it follows that $L[e^{rt}] = 0$ and $y = e^{rt}$ is a solution. The polynomial $Z(r)$ is called the \textbf{characteristic polynomial}, and the equation $Z(r) = 0$ is the \textbf{characteristic equation} of the differential equation. By factoring, we get
    $$Z(r) = a_0(r - r_1)(r - r_2)\dots(r-r_n)$$
    where $r_1,r_2,\dots,r_n$ are the zeroes, some of which may be equal.
    \newline
    \textbf{Real and Unreal Roots.} If the roots of the characteristic equation are real and no two are equation, then we have $n$ distinct solutions $e^{r_1t}, e^{r_2t},\dots,e^{r_nt}$. If these functions are linearly independent, then the general solution is
    $$y = \sum_{i=0}^n c_ie^{r_it}$$
    If the Wronskian determinant is non zero, then they are linearly independent.
    \newline
    \textbf{Complex Roots.} If the characteristic equation has complex roots, they must occur in conjugate pairs, $\lambda \pm i\mu$, since the coefficient $a_0, a_1,\dots, a_n$ are real numbers. The general solution is still of the same form, but we can replace $e^{(\lambda + i\mu)t}$ and $e^{(\lambda - i\mu)t}$ by
    $$e^{\lambda t}\cos\mu t, \quad e^{\lambda t}\sin \mu t$$
    \textbf{Repeated Roots.} If the roots of the characteristic equation are not distinct, then the solution is not as clear. If a root of $Z(r) = 0$, say $r = r_1$ has a multiplicity $s$ (occurs $s$ times), then
    $$e^{r_1t}, te^{r_1t}, t^2e^{r_1t},\dots,t^{s-1}e^{r_1t}$$
    are corresponding solutions. For a complex root, every time $\lambda + i\mu$ is repeated, $\lambda - i\mu$ must also repeat.
