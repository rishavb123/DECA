\section{Classifications of Differential Equations}
    If a function depends on a single variable, and then you take the derivative with respect to that variable, you are taking an ordinary derivative.
    \begin{equation*}
        f(x) = 2x^2
    \end{equation*}
    \begin{equation*}
        \frac{df}{dx} = 4x
    \end{equation*}
    If a function depends on multiple variables, and you are taking a derivative with respect to one of them, then it is a partial derivative.
    \begin{equation*}
        f(x, y) = 2xy
    \end{equation*}
    \begin{equation*}
        \frac{\partial f}{\partial x} = 2y
    \end{equation*}
    An equation with only ordinary derivatives is an \textbf{ordinary differential equation}, while an equation with both ordinary and partial derivatives is a \textbf{partial differential equation}.
    \newline \newline
    If there are more than one unknown functions that you are looking for, use a system of differential equations.
    \newline \newline
    The \textbf{order} of a differential equation is the order of the highest derivative that appears in the equation. Using $y = u(t)$
    \begin{equation*}
        F(t, u(t), u'(t), u''(t), \dots, u^{(n)}(t)) = F(t, y, y', y'', \dots, y^{(n)}) = 0
    \end{equation*}
    A differential equation is said to be \textbf{linear} if it can be written in the form
    \begin{equation*}
        a_0(t)y^{(n)} + a_1(t)y^{(n - 1)} + \dots + a_n(t)y = \sum_{i=0}^{n} a_i(t)y^{(n-i)} = g(t)
    \end{equation*}
    An equation that cannot be written in that form is called a \textbf{nonlinear} equation. For example,
    \begin{equation*}
        \frac{d^2\theta}{dt^2} + \frac{g}{L} \sin \theta = 0
    \end{equation*}
    Since the methods solving linear differential equations are highly developed, it is sometimes useful to use linear approximations of nonlinear functions. This process of approximating a nonlinear equation by a linear one is called \textbf{linearization}. For example, when $\theta$ is small, $sin \theta \approx \theta$. So the previous equation becomes
    \begin{equation*}
        \frac{d^2\theta}{dt^2} + \frac{g}{L} \theta = 0
    \end{equation*}
    A \textbf{solution} of an ordinary differential equation on the interval $\alpha < t < \beta$ is a function $\phi$ such that $\phi'$, $\phi''$, $\dots$, $\phi^{(n)}$ exist and satisy
    \begin{equation*}
        \phi^{(n)}(t) = f[t, \phi(t), \phi'(t), \phi''(t), \dots, \phi^{(n - 1)}(t)]
    \end{equation*}
    for every t in $\alpha < t < \beta$, $t \in \{r \in \textbf{R}: \alpha < t < \beta\}$, or $t \in (\alpha, \beta)$.