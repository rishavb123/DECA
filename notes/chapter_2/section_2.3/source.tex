\section{Modelling with First Order Equations}
    \textbf{Construction of the Model}. In this step you translate the physical situation into mathematical terms. Mathematical equations are almost always only an approximate description of the actual process. Sometimes you will conceptually replacement of a discrete process by a continuous one.
    \newline \indent
    \textbf{Analysis of the Model.} In this step, you are either solving the differential equation or finding out as many properties about it as possible. Sometimes further approximations help to solve this equation. THese approximations should be examined from a physical point of view so that it still reflects the physical features of the process.
    \newline \indent
    \textbf{Comparison with Experiment or Observation. } Now you interpret your solution or information in the context in which the problem arose. It should appear physically reasonable. If possible, check the solution with a known point.